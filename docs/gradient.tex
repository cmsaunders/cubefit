\documentclass[12pt]{article}
\usepackage{amsmath}
\usepackage{txfonts}

%- Abbreviations.
\newcommand{\psf}{\mathbf{H}}
\newcommand{\hhat}{\mathbf{\hat h}}
\newcommand{\x}{\mathbf{x}}
\newcommand{\xhat}{\mathbf{\hat x}}
\newcommand{\y}{\mathbf{y}}
\newcommand{\rvect}{\mathbf{r}}
\newcommand{\W}{\mathbf{W}}
\newcommand{\A}{\mathbf{A}}
\newcommand{\avect}{\mathbf{a}}
\newcommand{\F}{\mathbf{F}}



\newcommand{\ddx}{\frac{\partial}{\partial \mathbf{x}}}
\newcommand{\ddxi}{\frac{\partial}{\partial \mathbf{x}_{i}}}

\newcommand{\Nsn}{\textbf{XXX}}
\newcommand{\Nspec}{\textbf{YYY}}


\newcommand{\rem}[1]{\textbf{[\textsl{#1}]}}
\newcommand{\szf}[1]{\textbf{\textsl{SZF: #1}}}

% -----------------------------------------------------------------------------
% Document

\begin{document}

\title{CubeFit $\chi^2$ gradient calculation}
\author{S.~Bongard, Kyle Barbary}
\maketitle
\begin{abstract}
In CubeFit, one of the fitting steps is to fit the galaxy model to the
data while holding other parameters (such as positions) fixed. The
galaxy model has on order $\sim$1 million parameters. The ability to
analytically calculate the gradient (a vector of length ~$\sim$ 1
million) on the fit $\chi^2$ is essential in order to do the fit
efficiently. This note explains the derivation of the analytic
gradient.
\end{abstract}

\section{Matrix representation of the discrete Fourier transform}

The discrete Fourier transform of a 1-d array $x_n$ is

\begin{equation}
X_k = \sum_{n=0}^{N-1} x_n e^{-i 2 \pi k n / N}
\end{equation}

We can write this in matrix form,

\begin{equation}
\mathbf{X} = \left( \begin{array}{cccc}
   e^{-i 2\pi 0 \cdot 0 / N} & e^{-i 2\pi 0\cdot 1 / N} & e^{-i 2\pi 0 \cdot 2 / N} & \cdots \\
   e^{-i 2\pi 1 \cdot 0 / N} & e^{-i 2\pi 1 \cdot1 / N} & e^{-i 2\pi 1 \cdot 2 / N} & \cdots \\
   e^{-i 2\pi 2 \cdot 0 / N} & e^{-i 2\pi 2 \cdot 1 / N} & e^{-i 2\pi 2 \cdot 2 / N} & \cdots \\
   \vdots & \vdots & \vdots & \ddots
   \end{array} \right)
   \left( \begin{array}{c} x_0 \\ x_1 \\ x_2 \\ \vdots \end{array} \right)
\end{equation}

For a shorthand, we will call the square matrix $\mathbf{F}$

\begin{equation}
\mathbf{X} = \mathbf{F} \mathbf{x}
\end{equation}

Similarly, the inverse transform can be written

\begin{equation}
x_n = \frac{1}{N} \sum_{k=0}^{N-1} X_k e^{i 2 \pi k n / N}
\end{equation}

or

\begin{equation}
   \mathbf{x} = \frac{1}{N} \left( \begin{array}{cccc}
   e^{i 2\pi 0 \cdot 0 / N} & e^{i 2\pi 1 \cdot 0 / N} & e^{i 2\pi 2 \cdot 0 / N} & \cdots \\
   e^{i 2\pi 0 \cdot 1 / N} & e^{i 2\pi 1 \cdot 1 / N} & e^{i 2\pi 2 \cdot 1 / N} & \cdots \\
   e^{i 2\pi 0 \cdot 2 / N} & e^{i 2\pi 1 \cdot 2 / N} & e^{i 2\pi 2 \cdot 2 / N} & \cdots \\
   \vdots & \vdots & \vdots & \ddots
   \end{array} \right)
   \left( \begin{array}{c} X_0 \\ X_1 \\ X_2 \\ \vdots \end{array} \right)
\end{equation}

or as shorthand,

\begin{equation}
\mathbf{x} = \mathbf{F}^{-1} \mathbf{X}
\end{equation}

Now, note that the matrix $\mathbf{F}$ has the properties

\begin{equation}
\mathbf{F}^{-1} = \frac{1}{N} \bar{\mathbf{F}}
\end{equation}

where $\bar{\mathbf{F}}$ is the element-wise complex
conjugate of $\mathbf{F}$.


\section{Matricial form}

\begin{equation}
  \label{eq:1}
  \ddx \x^{T}\psf\x = (\psf + \psf^{T}) \x
\end{equation}

\begin{eqnarray*}
  \label{eq:2}
  \ddxi \x^{T}\psf\x & = & \ddxi \left( \sum_{j} x_{j} \sum_{l} H_{j,l} x_{l}  \right) \\
                     & = & \sum_{l} H_{i,l} x_{l} + \sum{j} H_{j,i} x_{j} \\
                     & = & \left( \psf \x + \psf^{T} \x␇ \right)_{i}
\end{eqnarray*}


By writing explicitly $\partial (\psf \x)_{j} / \partial x_{i}
= \partial/\partial x_{i} \left( \sum_{l} H_{j,l} x_{l}\right)$ and its transpose
it is easy to show the two following formulae. It is just important to make sure
to write the product in the exact same way depending on if $\x$ is on the left
or on the right of $\psf$: $\left( \psf \x \right)_{j} =  \sum_{l} H_{l,j} x_{l}$
and $\left( \x^{T} \psf \right)_{j} =  \sum_{l} x_{l} H_{j,l}$

\begin{equation}
  \label{eq:3}
  \ddx \left( \x^{T} \psf \right) = \ddx \left( \x^{T} \psf \right)^{T} = \psf^{T}
\end{equation}

\begin{equation}
  \label{eq:4}
  \ddx \left( \psf \x \right) =  \ddx \left( \psf \x \right)^{T} = \psf
\end{equation}



From now on, we assume that $\W = \W^{T}$.

\begin{eqnarray*}
  \label{eq:5}
  \ddx \chi^{2} & = & \ddx \left( (\psf \x - \y)^{T} \W (\psf \x - y) \right)\\
                & = & \ddx \left( (\psf \x)^{T} \W \psf \x \right)  
                    - \ddx \left( (\psf \x)^{T} \W \y \right)
                    - \ddx \left( \y^{T} \W \psf \x \right) \\
                & = & \ddx \left( \x^{T} \psf^{T} \W \psf \x \right)
                    - \ddx \left( \x^{T} \psf^{T} \W \y \right)
                    - \ddx \left( \y^{T} \W \psf \x \right) \\
                & = & \ddx \left( \x^{T} \psf^{T} \W \psf \x \right)
                    - \ddx \left( \x^{T} \psf^{T} \W \y \right)
                    - \ddx \left( (\psf^{T} \W \y)^{T} \x \right) \\
\end{eqnarray*}

We note that $(\psf^{T} \W \y)^{T}$ and $\psf^{T} \W \y$ are vectors. The two
last terms of the equation are thus of the form $\frac{\partial \x^{T}
  \mathbf{a}}{\partial \x} = \frac{\partial \mathbf{a}^{T} \x}{\partial \x} =
\mathbf{a} $, with $\mathbf{a} = \psf^{T} \W \y$ . And thus finally we have:

\begin{eqnarray*}
  \label{eq:6}
  \ddx \chi^{2} & = & \left( \psf^{T} \W \psf + (\psf^{T} \W \psf)^{T} \right) 
                       - 2 \psf^{T} \W \y\\
                & = & 2 \psf^{T} \W \psf \x - 2 \psf^{T} \W \y \\
                & = & 2 \psf^{T} \W \left( \psf \x - \y \right) \\
\end{eqnarray*}


\section{Fourier transform form}

The discrete Fourier transform of a vector $\x$ is:

\begin{equation}
  \label{eq:7}
  F(\x)_{k} = \sum_{j}^{N} x_{j} e^{- \frac{2\pi k j}{N}} = (\F \x)_{k}
\end{equation}

And the inverse discrete Fourier transform of vector $\xhat$ is
\begin{equation}
  \label{eq:8}
  F(\xhat)_{j} = \frac{1}{N} \sum_{k}^{N} \hat x_{k} e^{ \frac{2\pi j k}{N}} =
  (\F^{-1} \xhat)_{j}
\end{equation}

We chose to use the $1/N$ factor in the inverse transform because it makes the
normalization easier to carry accros calculations.

Since we can switch $j$ and $k$ in the exponentials above, we see that the
discrete direct and inverse Fourier transform in matricial form are related by
the formulae:
\begin{eqnarray*}
  \label{eq:9}
  \F^{-1} & = &\frac{1}{N} \F^{*T} \\
  \F^{-1*T} & = &\frac{1}{N} \F \\
\end{eqnarray*}
where $^{*}$ denotes the complex conjugation.


Let us now introduce the $diag()$ operator, which transforms a vector into a
diagonal matrix, the diagonal being the vector it operates upon: if $\A =
diag(\avect)$, then $\A_{i,j} = \delta_{i,j} a_{i}$. The vector obtained by an
element to element multiplication between $\x$ and $\avect$ can thus be writen
$diag(\avect) \x$.

In matrix form, the Fourier transform of the PSF convolution of vector $\x$ thus
writes as the element to element product of $\hhat = \F(\psf)$ and $\xhat =
\F(\x)$. 

The operator \emph{convolution by the PSF} thus writes in matricial form:
\begin{equation}
  \label{eq:11}
  \psf = \F^{-1} diag(\hhat) \F
\end{equation}

From the equation of the matricial derivative of the $\chi^{2}$ derived in the
previous section, we also need to now how to write $\psf^{T} = \F^{T}
diag(\hhat)^{T} \F^{-1T}$ in a form that has the Fourier transforms in the same
order. 

Since $\psf$ is a Real operator, we have $\psf = \psf^{*}$, and thus $\psf^{T} =
\psf^{*T}$.

Therefore, using the relations between direct and indirect Fourier transposes
derived above and the fact that the transpose of a diagonal matrix is the matrix
itself, we find:
\begin{eqnarray*}
  \label{eq:12}
  \psf^{T} & = & \F^{T} diag(\hhat)^{T} \F^{-1T}\\
           & = & (\F^{T} diag(\hhat)^{T} \F^{-1T})^{*}\\
           & = & \F^{T*} diag(\hhat)^{T*} \F^{-1T*}\\
           & = & N\F^{-1} diag(\hhat)^{*} \frac{1}{N}\F\\
           & = & \F^{-1} diag(\hhat)^{*} \F\\
\end{eqnarray*}

If we now write the residual $\rvect = \psf \x - \y$, we finally find that the
derivative of the $\chi^{2}$ writes:
\begin{equation}
  \label{eq:13}
  \ddx \chi^{2} = \F^{-1} diag(\hhat)^{*} \F(\W \mathbf{\rvect}) 
\end{equation}







\end{document}

