\documentclass[12pt]{article}
\usepackage{amsmath}
\usepackage{txfonts}

%- Abbreviations.
\newcommand{\psf}{\mathbf{H}}
\newcommand{\G}{\mathbf{G}}
\newcommand{\p}{\mathbf{p}}
\newcommand{\phat}{\mathbf{\hat p}}
\newcommand{\hhat}{\mathbf{\hat h}}
\newcommand{\x}{\mathbf{x}}
\newcommand{\xhat}{\mathbf{\hat x}}
\newcommand{\y}{\mathbf{y}}
\newcommand{\rvect}{\mathbf{r}}
\newcommand{\W}{\mathbf{W}}
\newcommand{\A}{\mathbf{A}}
\newcommand{\avec}{\mathbf{a}}
\newcommand{\F}{\mathbf{F}}
\newcommand{\V}{\mathbf{V}}
\newcommand{\diag}{\mathrm{diag}}

\newcommand{\ddx}{\frac{\partial}{\partial \mathbf{x}}}
\newcommand{\ddxi}{\frac{\partial}{\partial x_i}}

\newcommand{\Nsn}{\textbf{XXX}}
\newcommand{\Nspec}{\textbf{YYY}}


\newcommand{\rem}[1]{\textbf{[\textsl{#1}]}}
\newcommand{\szf}[1]{\textbf{\textsl{SZF: #1}}}

% -----------------------------------------------------------------------------
% Document

\begin{document}

\title{CubeFit $\chi^2$ gradient calculation}
\author{S.~Bongard, Kyle~Barbary, Clare~Saunders}
\maketitle
\begin{abstract}
In CubeFit, one of the fitting steps is to fit the galaxy model to the
data while holding other parameters (such as positions) fixed. The
galaxy model has on the order of one million parameters. The ability to
analytically calculate the gradient (a vector of length $\sim$ 1
million) on the fit $\chi^2$ is essential in order to carry out the fit
efficiently. This note explains the derivation of the gradient function.
\end{abstract}


\section{The Gradient}

The $\chi^2$ in the galaxy fit is given by

\begin{equation}
\chi^2 = (\y - \G\x)^T \W (\y - \G\x)
\end{equation}

\noindent where $\y$ is the data, $\x$ is the model parameters, $\W$
is the weight matrix (inverse covariance matrix; diagonal in our case)
and $\G$ is the operator that translates the model into the data
frame. $\G$ includes the effect of PSF convolution, shift operation
and cropping. Note that we are supposing that we can represent all
these operations as a matrix $\G$, but we will not show this until a
later section. Remember also that this is all in real space.  Carrying
on,

\begin{eqnarray*}
  \label{eq:5}
  \ddx \chi^{2} & = & \ddx \left( (\y - \G\x)^{T} \W (\y - \G\x) \right)\\
                & = & \ddx \left( (\G \x)^{T} \W \G \x \right)
                    - \ddx \left( (\G \x)^{T} \W \y \right)
                    - \ddx \left( \y^{T} \W \G \x \right) \\
                & = & \ddx \left( \x^{T} \G^{T} \W \G \x \right)
                    - \ddx \left( \x^{T} \G^{T} \W \y \right)
                    - \ddx \left( (\G^{T} \W \y)^{T} \x \right)
\end{eqnarray*}

\noindent where in the second line we have simply separated terms, dropping the
constant term $\y^T \W \y$, and in the third line we have used the property
that $(\mathbf{A}\mathbf{B})^T = \mathbf{B}^T\mathbf{A}^T$ and assumed that
$\W = \W^T$ (the weight matrix is symmetric).

We note that $(\G^{T} \W \y)^{T}$ and $\G^{T} \W \y$ are vectors ($\y$ is a
vector). The first two terms of the equation are thus of the form
$\ddx \x^T \avec$ and $\ddx \avec^T \x$. These are both simply equal to
$\avec$ as $\x^T \avec$ and $\avec^T \x$ are dot products between the
constant vector $\avec$ and $\x$, so the partial derivative with respect to
$\x$ is just $\avec$. Making this substitution, we have

\begin{equation}
\ddx \chi^{2} = \ddx \left( \x^{T} \G^{T} \W \G \x \right) - 2 \G^T \W \y 
\end{equation}

\noindent The first term has the form $\ddx \x^T\A\x$, with
$\A = \G^T \W \G$. It helps to write this out:

\begin{eqnarray*}
  \ddxi \x^{T}\A\x & = & \ddxi \left( \sum_{j} x_{j} \sum_l A_{j,l} x_{l}  \right) \\
                   & = & \sum_l A_{i,l} x_{l} + \sum_j A_{j,i} x_{j} \\
                   & = & \left( \A \x + \A^{T} \x␇ \right)_{i}
\end{eqnarray*}

\noindent or without the $i$ subscripts,

\begin{equation}
  \ddx \x^T\A\x = (\A + \A^T) \x
\end{equation}

\noindent Substituting this in (with $\A = \G^T \W \G$), we finally get:

\begin{eqnarray*}
\ddx \chi^{2} & = & \left( \G^T \W \G \x + (\G^T \W \G)^T \x \right)
                 - 2 \G^T \W \y \\
             & = & 2 \G^T \W \G \x - 2 \G^T \W \y \\
             & = & -2 \G^T \W ( \y - \G \x ) \\
             & = & -2 \G^T \W \rvect \\
\end{eqnarray*}

\noindent where $\rvect$ is shorthand for the residual
$\y - \G \x$. But what is the matrix $\G$ and what is $\G^T$?
For that we need to digress about
Fourier transforms as matricies (in the next section).

\section{Discrete Fourier Transform: Matrix Form}

The discrete Fourier transform of a vector $x_n$ is

\begin{equation}
\hat{x}_k = \sum_{n=0}^{N-1} x_n e^{-i 2 \pi k n / N}
\end{equation}

\noindent We can write this in matrix form,

\begin{equation}
\xhat = \left( \begin{array}{cccc}
   e^{-i 2\pi 0 \cdot 0 / N} & e^{-i 2\pi 0\cdot 1 / N} & e^{-i 2\pi 0 \cdot 2 / N} & \cdots \\
   e^{-i 2\pi 1 \cdot 0 / N} & e^{-i 2\pi 1 \cdot1 / N} & e^{-i 2\pi 1 \cdot 2 / N} & \cdots \\
   e^{-i 2\pi 2 \cdot 0 / N} & e^{-i 2\pi 2 \cdot 1 / N} & e^{-i 2\pi 2 \cdot 2 / N} & \cdots \\
   \vdots & \vdots & \vdots & \ddots
   \end{array} \right)
   \left( \begin{array}{c} x_0 \\ x_1 \\ x_2 \\ \vdots \end{array} \right)
\end{equation}

\noindent For a shorthand, we will call the square matrix $\mathbf{F}$

\begin{equation}
\xhat = \F \x
\end{equation}

Similarly, the inverse transform can be written

\begin{equation}
x_n = \frac{1}{N} \sum_{k=0}^{N-1} \hat{x}_k e^{i 2 \pi k n / N}
\end{equation}

\noindent or

\begin{equation}
   \mathbf{x} = \frac{1}{N} \left( \begin{array}{cccc}
   e^{i 2\pi 0 \cdot 0 / N} & e^{i 2\pi 1 \cdot 0 / N} & e^{i 2\pi 2 \cdot 0 / N} & \cdots \\
   e^{i 2\pi 0 \cdot 1 / N} & e^{i 2\pi 1 \cdot 1 / N} & e^{i 2\pi 2 \cdot 1 / N} & \cdots \\
   e^{i 2\pi 0 \cdot 2 / N} & e^{i 2\pi 1 \cdot 2 / N} & e^{i 2\pi 2 \cdot 2 / N} & \cdots \\
   \vdots & \vdots & \vdots & \ddots
   \end{array} \right)
   \left( \begin{array}{c} \hat{x}_0 \\ \hat{x}_1 \\ \hat{x}_2 \\ \vdots \end{array} \right)
\end{equation}

\noindent or as shorthand,

\begin{equation}
\x = \F^{-1} \xhat
\end{equation}

\noindent Now, note that the matrix $\F$ has the property

\begin{equation}
\F^{-1} = \frac{1}{N} \bar{\F}
\end{equation}

\noindent where $\bar{\F}$ is the element-wise complex conjugate of $\F$.


\section{The $\G$ matrix }

The $\G$ matrix represents the operations applied to the model to put
it "in the data frame". These are (1) convolve model by PSF (2) shift
model to align with data (3) crop model indicies that extend past the
data (model is spatially larger than data). Convolution in real space
is equivalent to element-wise multiplication (the Hadamard product) in
Fourier space (and this is indeed how the convolution is implemented
in CubeFit). While element-wise multiplication can be written
$\mathbf{a} \circ \mathbf{b}$, we will find it useful to write it as
$\diag(\mathbf{a}) \mathbf{b}$, where diag() is an operator
that translates a vector into a square diagonal matrix with the same
entries.

Convolution of the galaxy model $\x$ by a real-space PSF $\p$ can thus be
written as
\begin{equation}
\F^{-1} \diag(\F \p) \F \x
\end{equation}

Similarly, spatial shifts are implemented as element-wise multiplication in
Fourier space. the key point being that we can
subsume this shift vector (or rather the Fourier transform of it) into
$\F \p = \phat$. Therefore, we can write $\G$ as

\begin{equation}
\G = \F^{-1} \diag(\phat) \F
\end{equation}

We are now interested in calculating $\G^T$ as it appears in the
equation for $\ddx \chi^{2}$. Simply taking the transpose, we get
$\G^{T} = \F^T \diag(\phat)^{T} \F^{-1T}$, but we would like
to rewrite this in a form that has the Fourier transforms in the same
order as in $\G$. Since $\G$ is a real operator, we have $\G =
\bar{\G}$, and thus $\G^T = \bar{\G}^T$. We will use this along with
the relation between the forward and reverse Fourier transform derived
above ($\F^{-1} = \frac{1}{N} \bar{F}$), and the fact that the
transpose of a diagonal matrix is the matrix itself. We find:

\begin{eqnarray*}
  \G^{T} & = & \F^{T} \diag(\phat)^{T} \F^{-1T}\\
        & = & \overline{\F^{T} \diag(\phat)^{T} \F^{-1T}} \\
        & = & \bar{\F}^T \overline{\diag(\phat)^T} \bar{\F}^{-1T} \\
        & = & N\F^{-1} \overline{\diag(\phat)} \frac{1}{N}\F \\
        & = & \F^{-1} \overline{\diag(\phat)} \F\\
\end{eqnarray*}

\noindent If we now write the residual $\rvect = \y - \G\x$, we
finally find that the derivative of $\chi^{2}$ is:

\begin{eqnarray*}
 \ddx \chi^2 & = & -2 \G^T \W \rvect \\
             & = & -2 \F^{-1} \overline{\diag(\phat)} \F \W \rvect
\end{eqnarray*}

\noindent In CubeFit, the operator $\G^T = \F^{-1}
\overline{\diag(\phat)} \F$ is currently implemented by the method
\verb|AtmModel.gradient_helper()|.

\section{Fitting galaxy and sky simultaneously}

In another step in CubeFit, we fit the galaxy model to multiple epochs
of data simultaneously, while adjusting the sky for each epoch on each
iteration.  On each iteration, the sky is calculated by taking the
weighted average of the residual ($\y - \G \x$) at each wavelength and
subtracting that to form the new residual. We need to take this into
account in the $\chi^2$ gradient: qualitatively, increasing any given
model pixel will result in a lower sky estimate, and the sky estimate
feeds back into the $\chi^2$. The actual $\chi^2$ we're using involves
the residual $\y - \G\x - \mathrm{avg}(\y - \G\x)$. where avg() is a
weighted average over wavelength. Fortunately, this weighted average
can be represented by a matrix multiply. For example,

\begin{equation}
  \left( \begin{array}{cccc}
   \frac{w_0}{w_0 + w_1} & \frac{w_1}{w_0 + w_1} & 0 & 0 \\
   \frac{w_0}{w_0 + w_1} & \frac{w_1}{w_0 + w_1} & 0 & 0 \\
   0   & 0   & \frac{w_2}{w_2 + w_3} & \frac{w_3}{w_2 + w_3} \\
   0   & 0   & \frac{w_2}{w_2 + w_3} & \frac{w_3}{w_2 + w_3} \\
   \end{array} \right)
   \left( \begin{array}{c} x_0 \\ x_1 \\ x_2 \\ x_3 \end{array} \right)
\end{equation}

\noindent yields a vector whose first two entries are the weighted
average across the first two entries of $\x$ and whose last two
entries are the weighted average across the last two entries of
$\x$. Call this matrix $\V$. The weighted average of the residual is then
$\V (\y - \G\x)$ and the $\chi^2$ for each epoch is

\begin{eqnarray*}
\chi^2 &=& (\y - \G\x - \V(\y - \G\x))^T \W (\y - \G\x - \V(\y - \G\x)) \\
       &=& ((I - \V)(\y - \G\x))^T \W ((I - \V)(\y - \G\x)) \\
       &=& (\y - \G\x)^T (I - \V)^T \W (I - \V) (\y - \G\x)
\end{eqnarray*}

\noindent where in the third line we have used the equality
$(\mathbf{A}\mathbf{B})^T = \mathbf{B}^T \mathbf{A}^T$. This is the
same as the original $\chi^2$ equation, but with $\W$ replaced by $(I
- \V)^T \W (I - \V)$. This is a constant, so our original result for the
$\chi^2$ gradient holds under this replacement:

\begin{eqnarray*}
\W \rvect &\to& (I - \V)^T \W (I - \V) \rvect \\
          &\to& (I - \V)^T \W \mathbf{r'} \\
          &\to& \W \mathbf{r'} - \V^T \W \mathbf{r'}
\end{eqnarray*}

\noindent where $\mathbf{r'} = (I - \V)(\y - \G\x)$ is the residual after
subtracting the ``sky.''

Practically, how do we calculate $\V^T \W \mathbf{r'}$? In other words, if
multiplication by $\V$ is equivalent to a weighted average by wavelength,
what is $\V^T$? In our minimal example from above, it would be

\begin{equation}
  \V^T \x
  = \left( \begin{array}{cccc}
    \frac{w_0}{w_0 + w_1} & \frac{w_0}{w_0 + w_1} & 0 & 0 \\
    \frac{w_1}{w_0 + w_1} & \frac{w_1}{w_0 + w_1} & 0 & 0 \\
    0   & 0   & \frac{w_2}{w_2 + w_3} & \frac{w_2}{w_2 + w_3} \\
    0   & 0   & \frac{w_3}{w_2 + w_3} & \frac{w_3}{w_2 + w_3} \\
    \end{array} \right)
    \left( \begin{array}{c} x_0 \\ x_1 \\ x_2 \\ x_3 \end{array} \right)
  = \left( \begin{array}{c}
    \frac{w_0}{w_0 + w_1} (x_0 + x_1) \\
    \frac{w_1}{w_0 + w_1} (x_0 + x_1) \\
    \frac{w_2}{w_2 + w_3} (x_2 + x_3) \\
    \frac{w_3}{w_2 + w_3} (x_2 + x_3)
    \end{array} \right)
\end{equation}

\noindent Looking at the result, we can see this is equivalent to, for
each wavelength, taking the ratio of the sum of the input vector and
the sum of $\W$ and multiplying the result by $\W$. In Python (where
we are representing $\W$ and $\x$ with 3-d arrays), $\V^T \W \mathbf{r'}$
would be

\begin{verbatim}
tmp = np.sum(w * rprime, axis=(1, 2)) / np.sum(w, axis=(1, 2))
vtwrprime = w * tmp[:, None, None]
\end{verbatim}

\noindent where $\W$ is \verb|w| and $\mathbf{r'}$ is \verb|rprime|, and the
full replacement for $\W \rvect$ would be:

\begin{verbatim}
w * rprime - vtwrprime
\end{verbatim}

\section{Fitting SN and Data Positions}

In the final step of CubeFit, we fit simultaneously the position of
the data with respect to the galaxy and the position of the SN with
respect to the galaxy. Here, we calculate the analytic gradient of the
$\chi^2$ function used in this fit.

\begin{eqnarray*}
\chi^2 &=& \sum_{t, \lambda} \sum_i w_i (d_i - (scene)_i)^2 \\
       &=& \sum_{t, \lambda} \sum_i w_i (d_i - (sky + g_i + SN s_i))^2
\end{eqnarray*}

\noindent where the first sum is over epochs and wavelengths and the
second sum is over pixels at a give wavelength. $g_i$ and $s_i$ depend
on the positions that we are optimizing, as do $sky$ and $SN$, which
are determined based on the galaxy model, $\vec{g}$ and SN model,
$\vec{s}$.  The derivative of $\chi^2$ with respect to the position
$x_j$ of the data is

\begin{eqnarray*}
\frac{d\chi^2}{dx_\ast} &=& \sum_{t, \lambda} \sum_i -2 w_i (d_i - (scene)_i) \frac{d(scene)_i}{dx_\ast} \\
                     &=& \sum_{t, \lambda} \sum_i -2 w_i (d_i - (scene)_i) \frac{d}{dx_\ast} (sky + g_i + SN s_i) \\
                     &=&  \sum_{t, \lambda} \sum_i -2 w_i (d_i - (scene)_i) \left(\frac{d(sky)}{dx_\ast} + \frac{dg_i}{dx_\ast} + \frac{d(SN)}{dx_\ast} s_i + SN \frac{ds_i}{dx_\ast}\right)
\end{eqnarray*}

\noindent If $x_\ast$ is an SN position parameter then $dg_i/dx_\ast = 0$,
but otherwise all terms above are nonzero: $sky$ and $SN$
depend on $x_\ast$ through their dependence on $\vec{g}$ and
$\vec{s}$. The derivatives $dg_i/dx_\ast$ and $ds_i/dx_\ast$ are fairly
simple to compute. $\vec{g}$ and $\vec{s}$ are computed via
multiplication by a phase shift array in Fourier space. The phase
shift array is the only part that depends on position and everything
else is just multiplication. So all we need to do is to subsitute the
derivative on the phase shift array, then carry out all calculations
as we normally would.

We must also determine the gradient in $sky$ and $SN$ with respect to
position.  We start by defining some quantities for
convenience (in the left column), and taking their derivatives with respect to
data and SN positions. (Note that $s_i$ is affected by both positions, $g_i$ is only affected by the data position.)

\begin{displaymath}
%denom = AC - B^2
\begin{array}{ccc}
A = \sum_i w_i s_i^2    & \frac{dA}{dx_\ast} = \sum_i 2 w_i s_i \frac{ds_i}{dx_\ast}   &   (x_\ast \in \{x_{data}, x_{sn}\}) \\
B = \sum_i w_i s_i    & \frac{dB}{dx_\ast} = \sum_i w_i \frac{ds_i}{dx_\ast} \\
C = \sum_i w_i         & \frac{dC}{dx_\ast} = 0 \\
D = \sum_i w_i d_i     & \frac{dD}{dx_\ast} = 0 \\% wd
E = \sum_i w_i d_i s_i  & \frac {dE}{dx_\ast} = \sum_i w_i d_i  \frac{ds_i}{dx_\ast} \\ % wdsn
F = \sum_i w_i g_i     & \frac{dF}{dx_{data}} = \sum_i w_i \frac{dg_i}{dx_{data}}   &    \frac{dF}{dx_{sn}} = 0 \\ % wgal
G = \sum_i w_i g_i s_i  & \frac{dG}{dx_{data}} = \sum_i w_i \left(g_i \frac{ds_i}{dx_{data}} + s_i \frac{dg_i}{dx_{data}} \right) & \frac{dG}{dx_{sn}} = \sum_i w_i g_i \frac{ds_i}{dx_{sn}} % wgalsn
\end{array}
\end{displaymath}

\noindent where $i$ ranges over all pixels at the given wavelength,
$s_i$ is the SN model (scaled to 1) $g_i$ is the galaxy model, $d_i$
is the data and $w_i$ is the weight of the data. The SN amplitude, a
scalar here, is given by

\begin{equation}
SN = (-DB + EC + FB - GC) / (AC - B^2)
\end{equation}

\noindent (Derivation not shown.) Taking the derivative with
respect to $x_\ast$,

\begin{eqnarray*}
\frac{d(SN)}{dx_\ast} &=& \frac{-D'B - DB' + E'C + EC' + F'B + FB' - G'C - GC'}{AC - B^2} + \\
                     & & (SN) \frac{A'C + AC' - 2BB'}{AC-B^2} \\
                     &=& \frac{-DB' + E'C + F'B + FB' - G'C}{AC - B^2} + (SN) \frac{A'C - 2BB'}{AC-B^2}
\end{eqnarray*}

Similar to the SN amlitude, the sky is given by

\begin{equation}
sky = (DA - EB - FA + GB) / (AC - B^2)
\end{equation}

\noindent The derivative is

\begin{eqnarray*}
\frac{d(sky)}{dx_\ast} &=& \frac{D'A + DA' - E'B - EB' - F'A - FA' + G'B + GB'}{AC - B^2} + \\
                   & &  + (sky) \frac{A'C + AC' - 2BB'}{AC-B^2} \\
                   &=& \frac{DA' - E'B - EB' - F'A - FA' + G'B + GB'}{AC - B^2} + (sky) \frac{A'C - 2BB'}{AC-B^2}
\end{eqnarray*}


\end{document}
